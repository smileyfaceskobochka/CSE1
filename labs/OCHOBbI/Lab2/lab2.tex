\documentclass[oneside,a4paper,14pt]{extarticle}
\usepackage[a4paper,letterpaper,top=20mm,bottom=20mm,left=30mm,right=10mm]{geometry}
\usepackage[utf8]{inputenc}
\usepackage[russian]{babel}
\usepackage{amsmath}
\usepackage{amssymb}
\usepackage{cmap}
\usepackage{indentfirst}
\usepackage{graphicx}

\begin{document}

$$ 
    \delta_{ij} =
    \begin{cases}
        1, & i=j,\\
        0, & i\ne j.
    \end{cases}
$$
$$\delta_{ij}=\begin{cases}1,&i=j\\0,&i\ne j\end{cases}$$
\noindent
Текст % А
будет не % комментарий
прерывен! % пропадет!
\AA \aa \bf\AA
\textbf\AA
\rule[5pt]{25pt}{1pt} \rule[-5pt]{1pt}{10pt} \rule{6pt}{6pt}\\
$x^2_ij \to x^2_{ij}$\\
\begin{itshape}
    Для верстки большого объема текста лучше пользоваться окружениями.
\end{itshape}
\textbf{при}мер\\
{Для начала создадим группу, вложенную в обычный текст.\it \large Здесь изменим шрифт на курсив большого размера. {\sf\small Создадим вложенную группу и установим рубленный шрифт малого размера.} Выйдем из вложенной группы. Восстановился большой курсив.}\\

\newpage
\title{Заголовок}
\author{Автор, или перечень авторов}
\date{Дата создания документа}
    \begin{abstract}
        Аннотация ...
    \end{abstract}
\maketitle

\section[О друзьях]{Слоны мои друзья!}\label{s:слоны}

\appendix
\section{Название}\label{}
Текст приложения\dots
\subsection{Заголовок}\label{}
Продолжение\dots

Хахахахах\dots\footnotemark[1]

\footnotetext[1]{Хахаха хихихи}
\end{document}